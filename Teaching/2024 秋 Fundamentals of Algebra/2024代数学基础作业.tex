\documentclass[a4paper,12pt]{article} 
\usepackage[top = 2.5cm, bottom = 2.5cm, left = 2.5cm, right = 2.5cm]{geometry} 
\usepackage[T1]{fontenc}
\usepackage[utf8]{inputenc}
\usepackage{amsmath}
\usepackage{booktabs}
\usepackage{graphicx} 
\usepackage{setspace}
\setlength{\parindent}{0in}
\usepackage{float}
\usepackage{circledsteps}
\usepackage{xcolor}
\usepackage{fancyhdr}
\usepackage{ctex}
\usepackage{pifont}
\usepackage{rotating}
\usepackage{amssymb}
\pagestyle{fancy} 
\fancyhf{} 


\begin{document}
 

\def\head{\begin{tabular}{p{15.5cm}} % This is a simple tabular environment to align your text nicely 
{\large \bf 代数学基础作业} \\
中国科学技术大学 \\ 2024 秋\quad 杨金榜  \\ 陈鉴 \& 王子涵 \& 辛雨 \& 张煜星\\
\hline % \hline produces horizontal lines.
\\
\end{tabular} % Our tabular environment ends here.

\vspace*{0.3cm}}% Now we want to add some vertical space in between the line and our title.
\head

\begin{center} % Everything within the center environment is centered.
	{\Large \bf 第一次作业} % <---- Don't forget to put in the right number
	\vspace{2mm}
	
        % YOUR NAMES GO HERE
	{\bf\quad   9/10 \quad  第二周/星期二} % <---- Fill in your names here!

\end{center}  




\begin{enumerate}

\item {\it (习题1.1) 对任意集合X,我们用$\rm id_X$表示X到自身的恒等映射。设$f:A\rightarrow B$是集合间的映射,A是非空集合。试证:
\begin{enumerate}
\item f为单射当且仅当存在$g:B\rightarrow A$,使得$g\circ f=\rm id_A$;
\item f为满射当且仅当存在$h:B\rightarrow A$,使得$f\circ h=\rm id_B$;
\item f为双射当且仅当存在唯一的$g:B\rightarrow A$,使得$f\circ g=\rm id_B,g\circ f=\rm id_A$。
\end{enumerate}
这里的g称为f的逆映射,通常也记为$f^{-1}$。证明双射的逆映射也是双射,并讨论逆映射与映射的原像集合之间的关系。}

\item {\it (习题1.2) 如果$f:A\rightarrow B,g:B\rightarrow C$均是一一对应,则$g\circ f:A\rightarrow C$也是一一对应,且$(g\circ f)^{-1}=f^{-1}\circ g^{-1}$。}

\item {\it (习题1.5) 设X是无限集,Y是X的有限子集。证明存在双射$X-Y\rightarrow X$。}

\item {\it (习题1.8) 设A,B是两个有限集合。
\begin{enumerate}
\item A到B的不同映射共有多少个?
\end{enumerate}}

\item {\it (习题1.9) 证明容斥原理(定理 1.24)。}

\item {\it (课堂练习1) 若$|A|=|B|<\infty,\forall f:A\rightarrow B$,f单$\iff$ f双$\iff$ f满。}

\item {\it (课堂练习2) 若$A\xrightarrow{f} B\xrightarrow{g} C\xrightarrow{h} D$,证明$h\circ (g\circ f)=(h\circ g)\circ f$。}

\end{enumerate}

\newpage

\begin{tabular}{p{15.5cm}} % This is a simple tabular environment to align your text nicely 
{\large \bf 代数学基础作业} \\
中国科学技术大学 \\ 2024 秋\quad 杨金榜  \\ 陈鉴 \& 王子涵 \& 辛雨 \& 张煜星\\
\hline % \hline produces horizontal lines.
\\
\end{tabular} % Our tabular environment ends here.

\vspace*{0.3cm} % Now we want to add some vertical space in between the line and our title.
\begin{center} % Everything within the center environment is centered.
	{\Large \bf 第二次作业} % <---- Don't forget to put in the right number
	\vspace{2mm}
	
        % YOUR NAMES GO HERE
	{\bf\quad   9/12 \quad  第二周/星期四} % <---- Fill in your names here!

\end{center}  

\begin{enumerate}\setcounter{enumi}{7}
\item {\it 回顾:设 $\varphi\colon K\times K \rightarrow K$ 为集合 $K$ 上的二元运算。若对任意$x,y,z\in K$都有$\varphi(\varphi(x,y),z)=\varphi(x,\varphi(y,z))$,则称 $\varphi$满足结合律。若存在$e\in K$使得对任意 $x\in K$ 都有 $\varphi(e,x)=x=\varphi(x,e)$, 则称 $e$ 为$\varphi$的单位元。若对任意$x,y\in K$都有 $\varphi(x,y)=\varphi(y,x)$, 则称$\varphi$满足交换律。
 
设 $K=\{A,B\}$ 为两个元素组成的集合, 请
\begin{enumerate}
\item 在集合$K$上找出所有的二元运算(用表格表示,共$16$个)  
\item 哪几个运算存在单位元
\item 那几个满足交换律
\item 共有$8$个满足结合律,尽可能多的找出它们.(不需要证明)
\end{enumerate}}


\item {\it \ding{172}验证:$\mathbb{Q}[i]$为域(在通常复数域上的$+,\cdot$运算下);
 \\ \ding{173}说明$K=\mathbb{Q}+\mathbb{Q}\sqrt[3]{2}=\{ a+b\sqrt[3]{2}|a,b\in \mathbb{Q}\}$在通常$+,\times$下不为域;
 \\ \ding{174}请试着构造$K$上两个二元运算$\oplus,\odot$,使得$(K,\oplus,\odot)$构成域(说明想法即可,无需证明)。}
 \item {\it 回顾: 设 $(K,+,\cdot)$为域. 任意元素$a\in K$存在负元, 我们将其记为 $-a$. 并将加法$b-a$定义为 $b+(-a)$. 任意非零元$a\in K\setminus{0}$存在逆元, 我们将其记为 $a^{-1}$. 并将除法$\frac{b}{a}$定义为 $b\cdot a^{-1}$. 请从域的公理体系出发证明:
 \[\frac{b}{a}-\frac{d}{c}=\frac{bc-ad}{ac}.\]}

 \item 证明数集$R=\left\{\left.\frac{x+y\sqrt{-3}}{2}\right| x\text{与}y\text{为同奇偶的整数}\right\}$在通常的加法和乘法下构成环.
 
 
 \item 记$\mathbb{R}[X]$为实系数多项式组成的集合,即
 \[\mathbb{R}[X]=\{a_0+a_1x+a_2x^2+\cdots+a_nx^n\mid 其中 n\in\mathbb{N} \ \text{以及} \ a_0,a_1,\cdots,a_n\in \mathbb{R} \}.\] 
 在这个集合上定义通常的多项式加法与乘法. 即, 对任意多项式 $f(x)=\sum\limits_{i=0}^{n}a_i x^{i}$ 和 $g(x)=\sum\limits_{i=0}^{m}b_i x^{i}$, 
 \[f(x)+g(x)=\sum_{i=0}^{\max\{m,n\}}(a_i+b_i) x^{i} \]
 \[f(x)\cdot g(x)=\sum_{k=0}^{m+n} \left(\sum_{i+j=k\atop i,j\geq 0}a_i b_j\right) x^{k}\]
 其中,若$i>n$,记$a_i=0$;若$j>m$,记$b_j=0$.
 则$(\mathbb{R}[X],+,\cdot)$ 构成环,称为\textcolor{blue}{实系数多项式环},简记为$\mathbb{R}[X]$. 
 \begin{enumerate}
 \item 请写出实系数多项式环中的零元,幺元以及负元.
 \item 请证明实系数多项式环是交换环.
 \item (附加,不写不影响作业评分) 请问实系数多项式环是否为域,为什么? 如果不是,能不能将集合$\mathbf{R}[X]$扩大,使得其在通常加法和乘法下构成域.
 \end{enumerate}
\end{enumerate}

\newpage
\head






\begin{center} % Everything within the center environment is centered.
	{\Large \bf 第三次作业} % <---- Don't forget to put in the right number
	\vspace{2mm}
	
        % YOUR NAMES GO HERE
	{\bf\quad   9/19 \quad  第三周/星期四} % <---- Fill in your names here!

\end{center} 




\begin{enumerate}\setcounter{enumi}{12}
    \item 回顾: 实数域上的二阶矩阵组成的集合为
 \[M_2(\mathbb R)=\left\{\left.\begin{pmatrix} a & b \\ c & d\\\end{pmatrix}\right|a,b,c,d\in\mathbb R\right\}.\]
 这个集合上可以定义如下加法和乘法运算:
 \[\begin{pmatrix} a & b \\ c & d\\\end{pmatrix} + \begin{pmatrix} a' & b' \\ c' & d'\\\end{pmatrix} = \begin{pmatrix} a+a' & b+b' \\ c+c' & d+d' \\\end{pmatrix} \]
 \[\begin{pmatrix} a & b \\ c & d\\\end{pmatrix} \cdot \begin{pmatrix} a' & b' \\ c' & d'\\\end{pmatrix} = \begin{pmatrix} aa'+bc' & ab'+bd' \\ ca'+dc' & cb'+dd' \\\end{pmatrix} \]
 则 $(M_2(\mathbb{R}),+,\cdot)$ 构成环,称为\textcolor{blue}{实数域上的二阶矩阵代数}. 
 \begin{enumerate}
 \item 证明矩阵代数上的乘法结合律.
 \item 举例说明矩阵代数上,乘法不满足消去律. 即由$ac=bc$以及$c\neq \begin{pmatrix}0&0\\0&0\\\end{pmatrix}$ 推不出$a=b$. 提示:求$\begin{pmatrix}0&1\\0&0\\\end{pmatrix}^2$.
 \end{enumerate}
\item 回顾: 复数域上的二阶矩阵组成的集合为
 \[M_2(\mathbb C)=\left\{\left.\begin{pmatrix} a & b \\ c & d\\\end{pmatrix}\right|a,b,c,d\in\mathbb C\right\}.\]
 这个集合上可以定义如下加法和乘法运算:
 \[\begin{pmatrix} a & b \\ c & d\\\end{pmatrix} + \begin{pmatrix} a' & b' \\ c' & d'\\\end{pmatrix} = \begin{pmatrix} a+a' & b+b' \\ c+c' & d+d' \\\end{pmatrix} \]
 \[\begin{pmatrix} a & b \\ c & d\\\end{pmatrix} \cdot \begin{pmatrix} a' & b' \\ c' & d'\\\end{pmatrix} = \begin{pmatrix} aa'+bc' & ab'+bd' \\ ca'+dc' & cb'+dd' \\\end{pmatrix} \]
 则 $(M_2(\mathbb C),+,\cdot)$ 构成环. 此外, 我们对任意 $e\in\mathbb C$ 我们定义如下记号:
 \[e\begin{pmatrix} a & b \\ c & d\\\end{pmatrix} = \begin{pmatrix} ea & eb \\ ec & ed\\\end{pmatrix}.\]
 环$(M_2(\mathbb C),+,\cdot)$称为{\color{blue}{复数域上的二阶矩阵代数}}.
 记 
 \[i=\begin{pmatrix} \sqrt{-1} & 0 \\ 0 & -\sqrt{-1} \\ \end{pmatrix}, \quad j=\begin{pmatrix} 0 & 1 \\ -1 & 0 \\ \end{pmatrix}, \quad \text{以及} \quad k=\begin{pmatrix} 0 & \sqrt{-1} \\ \sqrt{-1} & 0 \\ \end{pmatrix}.\]
 证明:
 \begin{enumerate}
 \item $i^2=j^2=k^2=ijk=\begin{pmatrix}-1&0\\0&-1\end{pmatrix}$;
 \item $ij=-ji=k$; $jk=-kj=i$; $ki=-ik=j$;
 \item $M_2(\mathbb C)$的子集 $\mathbb H =\{a+bi+cj+dk\mid a,b,c,d \in \mathbb R\}\subset M_2(\mathbb C)$(这里的$a$视为$aI_2$,其中$I_2=\begin{pmatrix}
     1 & 0\\
     0 & 1
 \end{pmatrix}$) 关于乘法封闭;
 \end{enumerate}
 注: 实际上, $\mathbb H$构成$M_2(\mathbb C)$的子环,且其中任意非零元都有乘法逆元. 
 \item 给定集合$M$, 令$S_M$为$M$到自身的双射的集合. 证明若以映射的复合$\circ$作为$S_M$上的乘法, 则$\left(S_M,\circ\right)$构成一个群.
 \item 设集合$\mathbb{Z}[\sqrt{2}]=\{a+b\sqrt{2}\mid a,b\in\mathbb{Z}\}$, 验证它在实数加法和乘法意义下构成环.
 \item 设A为含幺非交换环,$a,b\in A$. 如果$ba=1$, 则称$b$为$a$的左逆, $a$为$b$的右逆. 如果并思考以下几个问题:
 \begin{enumerate}
     \item 如果$a$的左逆与右逆同时存在,则左逆等于右逆;
     \item 如果a的左逆存在且唯一,则a有右逆; (提示: 若$b$为$a$的左逆,则$ab+b-1$也为$a$的左逆.)
     \item 如果$a$的左逆不止一个,则必有无数个左逆. (提示:采用反证法. 考察由$a$的全体左逆组成的集合$Inv_a:=\{x\in A\mid xa=1\}$. 假若$Inv_a$有限且个数大于$1$,不妨设$x_1\in Inv_a$,则$\{1-ax+x_1\mid x\in Inv_a\}=Inv_a$). 
     %此外验证$x_1\not\in Inv_a$,因此矛盾!).
     \item {\color{red} (选做)} 请构造一个环$A$使得,里面存在元素$a$,其仅有左逆而没有右逆;
 \end{enumerate}
 \end{enumerate}

 {\color{red} 注意:9.19和9.24作业应于9.26-9.30期间提交}

 \newpage
 \head

 \begin{center} % Everything within the center environment is centered.
	{\Large \bf 第四次作业} % <---- Don't forget to put in the right number
	\vspace{2mm}
	
        % YOUR NAMES GO HERE
	{\bf\quad   9/24 \quad  第四周/星期二} % <---- Fill in your names here!

\end{center} 

\begin{enumerate}\setcounter{enumi}{17}
    \item 令集合$G=\left\{A,B\right\}$. 用表格的形式列出全部$G$上的运算$\varphi$, 使得$\left(G,\varphi\right)$构成
 \begin{itemize}
 \item[1)] 半群;
 \item[2)] 含幺半群;
 \item[3)] 群;
 \item[4)] 交换群.
 \end{itemize}
 \item 若$G$是群, $x,y\in G$, 定义$x,y$的换位子为
 $$\left[x,y\right]=xyx^{-1}y^{-1}.$$
 证明
 \begin{itemize}
 \item[1)] $\left[x,y\right]^{-1}=\left[y,x\right]$;
 \item[2)] $\left[xy,z\right]=x\left[y,z\right]x^{-1}\left[x,z\right]$;
 \item[3)] $\left[z,xy\right]=\left[z,x\right]x\left[z,y\right]x^{-1}$.
 \end{itemize}
 \item 令$\mu_{\infty}$为$\mathbb{C}$里的所有单位根(即$\mu_{\infty}=\{a\in \mathbb{C}|\text{存在正整数n使得} a^n=1\}$),证明$\mu_{\infty}$在复数乘法意义下构成群.
 \item 设A为集合,$P(A)$为$A$里的所有子集构成的集合,在$P(A)$上定义二元运算:$X\Delta Y=(X\bigcap Y^c)\bigcup (X^c \bigcap Y)$,证明$P(A)$在此运算下构成群.
 \item 我们给定一个乘法群G和其子集M:
 \begin{itemize}
     \item[1)] 我们定义$N_G(M)=\{g\in G| gMg^{-1}=M\}$,请证明$N_G(M)$是G的子群
     \item[2)] 我们定义$C_G(M)=\{g\in G| gag^{-1}=a,\forall a\in M\}$,请证明$C_G(M)$是G的子群
 \end{itemize}
 \item 设$G$为二元实数组构成的集合$\{(a,b)|a\neq 0\}$,我们定义$G$上的乘法为$(a,b)(c,d)=(ac,ad+b)$,求证G在此运算下是群.
 \item[] {\color{red}以下三题至少选做一题}
  \item 试着求出$S_3$的所有子群.\\
 试着求出$D_4$的所有子群.
 \item 设G是半群,若对任意的$a,b\in G$,方程$xa=b$和$ay=b$都在G里面有解,证明$G$是群.(提示: 若$ea=a$,则$eb=b$.)
 \item 设G是一个有限半群,如果在G内均有左右消去律成立,即由$ax=ay$或$xa=ya$都可以推得$x=y$,证明$G$是群. (提示:$G=\{ag\mid g\in G\}$.)
 
\end{enumerate}

{\color{red} 注意:9.19和9.24作业应于9.26-9.30期间提交}


\newpage
\head

 \begin{center} % Everything within the center environment is centered.
	{\Large \bf 第五次作业} % <---- Don't forget to put in the right number
	\vspace{2mm}
	
        % YOUR NAMES GO HERE
	{\bf\quad   9/26 \quad  第四周/星期四} % <---- Fill in your names here!

\end{center} 

 \begin{enumerate}\setcounter{enumi}{26}
     \item (习题2.14) 群$G$到自身的同构称为$G$的自同构。
     \begin{itemize}
         \item[1)] 证明群$G$的所有自同构在复合映射作为乘法下构成群。这个群称为$G$的自同构群,记为$AutG$;
         \item[2)] 如$\varphi:G\stackrel{\sim}{\longrightarrow}H$为群同构,证明$G$到$H$的所有同构构成集合$\varphi AutG := \{\varphi \circ f | f \in AutG\}$。
     \end{itemize}
     
     \item 设$G$是群。试问映射$x \rightarrow x^2$何时是群同态?并分别举例说明:这一映射可能是单同态但不是满同态,可能是满同态但不是单同态,也可能是同构。
     
     \item (习题2.16) 设$G$是群。证明映射$x \rightarrow x^{-1}$是群同构当且仅当$G$为阿贝尔群。
     
     \item (习题2.18) 证明乘法群$\mathbb{C}^{\times} \cong \mathbb{R}^{\times}_+ \times S^1$,其中$\mathbb{R}^{\times}_+$是正实数构成的乘法群。
     
     \item (习题2.25) 如果$I, J$均是交换环$R$的理想,证明
     $$ I + J = \{x + y | x \in I, y \in J\} $$
     与$I \cap J$都是$R$的理想。举例说明$I \cup J$不一定为$R$的理想。
     
     \item  回顾:正交群$O_2(\mathbb{R})$为:
      \[O_2(\mathbb R)=\left\{\left.\begin{pmatrix} cos \theta & \mp sin \theta \\ sin \theta & \pm cos \theta\\\end{pmatrix}\right|\theta\in\mathbb R\right\}.\]
     证明:
     \[O_2(\mathbb R)=\left\{\left.\begin{pmatrix} a & b \\ c & d\\\end{pmatrix}\right|\begin{pmatrix} a & b \\ c & d\\\end{pmatrix} \begin{pmatrix} a & c \\ b & d\\\end{pmatrix} = \begin{pmatrix} 1 & 0 \\ 0 & 1\\\end{pmatrix}, a,b,c,d\in\mathbb R\right\}.\]
     \newpage
     
     \item 设$(R, +, \cdot)$为环,$\varphi:R \rightarrow T$为双射。定义$T$上的二元运算$\oplus,\odot$:
     $$t_1 \oplus t_2 := \varphi(\varphi^{-1}(t_1) + \varphi^{-1}(t_2))$$
     $$t_1 \odot t_2 := \varphi(\varphi^{-1}(t_1) \cdot \varphi^{-1}(t_2))$$
     证明:
     \begin{itemize}
         \item[1)] $(T, \oplus, \odot)$构成环;
         \item[2)] $\varphi$是从$(R, +, \cdot)$到$(T, \oplus, \odot)$的环同构。
     \end{itemize}
     
     \item 求所有从$\mathbb{Q}[\sqrt{2}]$到$\mathbb{Q}[\sqrt{2}]$的环同态。
     
     \item {\color{red} (选做)}设$\varphi$是从$(\mathbb{R}, +, \cdot)$到$(\mathbb{R}, +, \cdot)$的环同构,
     试证明:$\varphi=id_{\mathbb{R}}$
 \end{enumerate}

{\color{red} 注意:9.26和9.29作业应于9.29-10.8期间提交}


\newpage
\head

 \begin{center} % Everything within the center environment is centered.
	{\Large \bf 第六次作业} % <---- Don't forget to put in the right number
	\vspace{2mm}
	
        % YOUR NAMES GO HERE
	{\bf\quad   9/29 \quad  第四周/星期日} % <---- Fill in your names here!
\end{center} 

  \begin{enumerate}\setcounter{enumi}{35}
     \item 证明映射$\varphi : \mathbb{C} \rightarrow M_2(\mathbb{R}), \ 
     a+b \sqrt{-1} \mapsto \begin{pmatrix} a & -b \\ b & a\\\end{pmatrix}$是环同态。

     \item 设$G_1, G_2$为群,在笛卡尔积$G_1 \times G_2$上定义乘法运算$\cdot$, 对任意$g_1,g_1'\in G_1$ 和 $g_2,g_2'\in G_2$,
     $$(g_1, g_2) \cdot (g_1', g_2') := (g_1g_1', g_2g_2').$$
     \begin{itemize}
         \item[1)] 证明$(G_1 \times G_2, \cdot)$构成群,称之为群$G_1$和群$G_2$的\textbf{直积}或者\textbf{笛卡尔积};
         \item[2)] 证明\textbf{投影映射}
         \[Pr_1:G_1 \times G_2 \rightarrow G_1, \ (g_1, g_2) \mapsto g_1\]
         和
         \[Pr_2:G_1 \times G_2 \rightarrow G_2, \ (g_1, g_2) \mapsto g_2\]
         均是群的满同态;
         \item[3)] 证明映射
         \[I_1:G_1 \rightarrow G_1 \times G_2, \ g_1 \mapsto (g_1, 1_{G_2})\]
         和
         \[I_2:G_2 \rightarrow G_1 \times G_2, \ g_2 \mapsto (1_{G_1}, g_2)\]
         均是群的单同态。
     \end{itemize}

     \item 设$R_1, R_2$为环,在笛卡尔积$R_1 \times R_2$上定义两个二元运算$+$和$\cdot$, 对任意$r_1,r_1'\in R_1$ 和 $r_2,r_2'\in R_2$,
     $$(r_1, r_2) + (r_1', r_2') := (r_1+r_1', r_2+r_2')$$
     $$(r_1, r_2) \cdot (r_1', r_2') := (r_1r_1', r_2r_2')$$
     \begin{itemize}
         \item[1)] 证明$(R_1 \times R_2, +, \cdot)$构成环,称之为环$R_1$与环$R_2$的\textbf{直积}或者\textbf{笛卡尔积};
         \item[2)] 证明投影映射
         \[Pr_1:R_1 \times R_2 \rightarrow R_1, \ (r_1, r_2) \mapsto r_1\]
         和
         \[Pr_2:R_1 \times R_2 \rightarrow R_2, \ (r_1, r_2) \mapsto r_2\]
         均是环的满同态;
         \item[3)] 设$R_1$与$R_2$都不为零环,证明映射
         \[I_1:R_1 \rightarrow R_1 \times R_2, \ r_1 \mapsto (r_1, 0_{R_2})\]
         和
         \[I_2:R_2 \rightarrow R_1 \times R_2, \ r_2 \mapsto (0_{R_1}, r_2)\]
         均保持加法和乘法但不是环同态;
         \item[4)] 证明若$R_1$与$R_2$都不为零环,则$R_1 \times R_2$一定不是整环。
     \end{itemize}

     \item 设$R$为交换环,证明:
     \begin{itemize}
         \item[1)] $R$为整环$\iff R[x]$为整环;
         \item[2)] 若$R$为整环,证明$R^{\times}=R[x]^{\times}$。
     \end{itemize}

     \item 设$R$为交换环,对任意$ a \in R$,定义$\varphi_a$为:$\varphi_a: R[x] \rightarrow R, \  f(x) \mapsto f(a)$. 证明$\varphi_a$是环的满同态,称之为\textbf{赋值映射}。

     \item 设$R$为交换环, $f,g \in R[x]$. 证明:
     \begin{itemize}
         \item[1)] $\deg(f+g) \leq \max\{\deg(f), \deg(g)\}$;
         \item[2)] $\deg(fg) \leq \deg(f)+\deg(g)$;
         \item[3)] 若$R$为整环,则$\deg(fg) = \deg(f)+\deg(g)$。
     \end{itemize}

     \item {\color{red} (选做)}证明:有限整环是域。
     
 \end{enumerate}

{\color{red} 注意:9.26和9.29作业应于9.29-10.8期间提交}


\newpage
\head

 \begin{center} % Everything within the center environment is centered.
	{\Large \bf 第七次作业} % <---- Don't forget to put in the right number
	\vspace{2mm}
	
        % YOUR NAMES GO HERE
	{\bf\quad   10/8 \quad  第六周/星期二} % <---- Fill in your names here!
\end{center} 

\begin{enumerate}\setcounter{enumi}{42}
        \item 设 $n$ 为正整数, 证明 $\text{gcd}(n!+1,(n+1)!+1)=1$. 这里 $n!=n(n-1)\cdots1$ 是 $n$ 的阶乘 .
   
        \item 设 $n$ 为正整数, $m$ 为正奇数 . 证明: $\text{gcd}(2^m-1,2^n+1)=1$. 
   
        \item 求 $\text{gcd}(1573,-1859),\text{gcd}(-121,-169),\text{gcd}(76501,9719)$.
   
        \item 设 $n$ 为正整数. 证明: $n$ 至多有 $2\lfloor\sqrt{n}\rfloor$ 个正因子. 这里$\lfloor ·\rfloor$ 表示向下取整 .
   
        \item 设 $n$ 为正整数 . 证明: $n^2|(n+1)^n-1$.
   
        \item  {\color{red} (选做)} 记 $X=\{m+\frac{n}{n+1}|n,m\in\mathbb{N}\}$. 证明: $X$ 的任意非空子集均有最小元, 即 $X$ 为良序集 . (注: 这里的序关系是继承自 $(\mathbb{R},\leq)$ 的)
\end{enumerate}

{\color{red} 注意:10.8和10.10作业应于10.10-10.15期间提交}

\newpage
\head

 \begin{center} % Everything within the center environment is centered.
	{\Large \bf 第八次作业} % <---- Don't forget to put in the right number
	\vspace{2mm}
	
        % YOUR NAMES GO HERE
	{\bf\quad   10/10 \quad  第六周/星期四} % <---- Fill in your names here!
\end{center} 

\begin{enumerate}\setcounter{enumi}{48}
    \item 设$n$为正整数,$a,b$ 为正整数,证明:\\
    (1) $\gcd(a^n,b^n)=\gcd(a,b)^n$;\\
    (2) 设$a,b$是互素的正整数,$c$ 为正整数,$ab=c^n$,则$a,b$都是某个正整数的$n$次方幂。
    \item 用欧几里得算法求$963$和$657$的最大公约数,并求出方程$963x+657y=\gcd(963,657)$的一组特解,以及所有整数解。
    \item 设$a,b$为正整数且$\gcd(a,b)=1$。证明:当整数$n>ab-a-b$时,方程$ax+by=n$有非负的整数解;但当$n=ab-a-b$时,该方程没有非负整数解。
    \item 求$\mathrm{lcm}(1573,-1859)$,$\mathrm{lcm}(-121,-169)$,$\mathrm{lcm}(76501,9719)$.

    \item  {\color{red} (选做)} 设 $p(x)=a_0+a_1x+\cdots+a_nx^n$ 为整系数多项式, $a_0,a_n\not=0$。证明:$p(x)$ 的有理数根 $x_0=\frac{p}{q}$ 满足 $p\mid a_0,q\mid a_n$, 其中 $p,q$ 为互素的整数。

    \item  {\color{red} (选做)} 求所有的正整数列 $\{a_i\}$ 满足 $\forall i\not=j, \gcd(i,j)=\gcd(a_i,a_j)$.(提示:Don't spend too much time on this question)
\end{enumerate}


{\color{red} 注意:10.8和10.10作业应于10.10-10.15期间提交}

\newpage
\head


 \begin{center} % Everything within the center environment is centered.
	{\Large \bf 第九次作业} % <---- Don't forget to put in the right number
	\vspace{2mm}
	
        % YOUR NAMES GO HERE
	{\bf\quad   10/15 \quad  第七周/星期二} % <---- Fill in your names here!
\end{center} 

\begin{enumerate}\setcounter{enumi}{54}

    \item 设$\varphi:R_1 \to R_2 $为环同态,证明:\\
    (1) 若$J\triangleleft R_2$,则$\varphi^{-1}(J):=\{r_1\in R_1|\varphi(r_1)\in J\}$为$R_1$的理想;\\
    (2) 若$I\triangleleft R_1$且$\varphi$为满射,则$\varphi(I)\triangleleft R_2$;\\
    (3) 给出反例说明若$\varphi$不为满射则(2)不一定成立。
    
    \item 设I为R的理想,证明:
    $$M_2(I):=\left\{\left. \begin{pmatrix} a & b \\ c & d\\\end{pmatrix}\right| a,b,c,d\in I\right\}$$
    为R上矩阵代数$M_2(R)$的理想。
    \item 设R为环(不一定交换),证明:
    $$(a)=\left\{\left.\sum_{i=1}^n x_iay_i\right|x_i.y_i\in R,n\in \mathbb{N}\right\},$$
    其中左边为由$a$生成的理想,即定义为包含$a$的最小理想。
    \item 设$R$为含幺交换环. 设$I_1,I_2\triangleleft R$是两个理想, 若$I_1+I_2=R$, 则称$I_1,I_2$\textbf{互素}.
 \begin{itemize}
 \item[(1)] 若$I_1,I_2$互素, 证明$I_1\cap I_2=I_1I_2$;
 \item[(2)] 若$I_1,\cdots,I_n$两两互素, 证明:
 \begin{itemize}
 \item[(a)] $I_1$与$I_2\cdots I_n$互素;
 \item[(b)] $I_1\cap\cdots\cap I_n=I_1\cdots I_n$.
 \end{itemize}
 \end{itemize}
\end{enumerate}


{\color{red} 注意:10.15和10.17作业应于10.17-10.22期间提交}

\newpage
\head

 \begin{center} % Everything within the center environment is centered.
	{\Large \bf 第十次作业} % <---- Don't forget to put in the right number
	\vspace{2mm}
	
        % YOUR NAMES GO HERE
	{\bf\quad   10/17 \quad  第七周/星期四} % <---- Fill in your names here!
\end{center} 

\begin{enumerate}\setcounter{enumi}{58}
 \item 设$f$为$\mathbb{Z}_{>0}$上的函数
 \begin{itemize}
 \item[(1)] 若对所有互素的正整数$m,n$, 有$f(mn)=f(m)f(n)$, 则称$f$为积性函数;
 \item[(2)] 若对任意的正整数$m,n$, 有$f(mn)=f(m)f(n)$, 则称$f$为完全积性函数;
 \end{itemize}
 设正整数$n$的因式分解为$n=p_1^{v_{p_1}(n)}\cdots p_s^{v_{p_s}(n)}$, 定义
 $$\sigma_k(n)=\sum_{d|n,d\geq 1}d^k$$
 证明:
 \begin{itemize}
 \item[(1)] $$\sigma_k(n)=\prod_{i=1}^s\frac{p_i^{(v_{p_i}(n)+1)k}-1}{p_i^k-1},(k\geq 1);$$
 \item[(2)] $\sigma_k(n)$为积性函数但不是完全积性函数.
 \end{itemize}
 \item (习题3.7) 设$n> 1$为整数,如果对于任何整数$m$,或者$n\mid m$或者$(n,m)=$
1,则 $n$ 必是素数.
 \item (习题3.8) 设整数$n>2$,证明:$n$和$n!$之间必有素数.由此证明素数有无穷
多个.
 \item (习题3.11) 设$a,b$是整数,$a\neq b,n$是正整数.如果$n\mid(a^n-b^n)$,则$n\mid\frac{a^n-b^n}{a-b}.$
 \item (习题3.12) 设$n\geqslant1.$证明:
  \begin{itemize}
\item[(1)] n为完全平方数的充要条件是$\sigma _0( n)$为奇数,
\item[(2)] $\sigma _0( n) \leqslant 2\sqrt {n}+ 1;$
\item[(3)] $n$的正约数之积等于$n^\frac{\sigma_0(n)}2.$
\end{itemize}
 \item (习题3.13) 设$m\in\mathbb{Z}_+$ 的因式分解为$m=\prod_ip_i^{\alpha_i}$.若$f$ 为积性函数,证明
$$f(m)=\prod_if(p_i^{\alpha_i}).$$
 \item {\color{red} (选做)} (习题3.9)
 \begin{itemize}
 \item[(1)]设$m$为正整数,证明:如果$2^m+1$为素数,则$m$为 2 的方幂。
\item[(2)]对$n\geqslant0$,记$F_n=2^{2^n}+1$,这称为费马数. 证明:如果$m>n$,则
$F_n\mid(F_m-2);$
\item[(3)]证明:如果$m\neq n$,则$(F_m,F_n)=1.$由此证明素数有无穷多个.
 \end{itemize}
 
 \item {\color{red} (选做)} (习题3.10)
  \begin{itemize}
 \item[(1)]设$m,n$都是大于 1 的整数,证明:如果$m^n-1$是素数,则
$m=2$并且$n$是素数.
\item[(2)]设$p$是素数,记$M_p=2^p-1$,这称为梅森数. 证明:如果$p,q$是不
同的素数,则$(M_p,M_q)=1.$
\end{itemize}
 \item {\color{red} (选做)} (习题3.14)对于 $n=p_1^{e_1}\cdots p_s^{e_s}\in\mathbb{Z}_+$,令


$$\mu\left(n\right)=\begin{cases}1,&\text{如果 }n=1,\\\left(-1\right)^{s},&\text{如果 }e_{1}=\cdots=e_{s}=1,\\0,&\text{其他情况}.\end{cases}$$


$\mu(n)$称为默比乌斯(Möbius)函数,证明:

$$\left.\sum_{1\leqslant d|n}\mu\left(d\right)=\left\{\begin{matrix}1,&\text{如果 }n=1,\\0,&\text{如果 }n>1.\end{matrix}\right.\right.$$
 \item {\color{red} (选做)} (习题3.15) 设$f(x)$和$g(x)$为两个定义在正整数集合$\mathbb{Z}_+$上的函数 (值域可以
为任何数域).证明:

\begin{itemize}
    \item[(1)] $g(n)=\sum_{1\leqslant d\mid n} f(d)$当且仅当$f(n)=\sum_{1\leqslant d\mid n}\mu(d)g(\frac n d).$
\item[(2)]如果$g\left(x\right)\neq0$,则$g(n)=\prod_{1\leqslant d\mid n}f(d) $当且仅当$f(n)=\prod_{1\leqslant d\mid n}g(\frac nd)^{\mu(d)}.$
\end{itemize}
其中$\mu$为上题的默比乌斯函数。上面两个等价关系习惯上称为 默比乌斯反演公式(Möbius inversion formula).

 \item  {\color{red} (选做)} 证明ED(欧几里得整环)$\Rightarrow$ PID(主理想整环)。
 
\end{enumerate}


{\color{red} 注意:10.15和10.17作业应于10.17-10.22期间提交}

\newpage
\head

\begin{center} % Everything within the center environment is centered.
	{\Large \bf 第十一次作业} % <---- Don't forget to put in the right number
	\vspace{2mm}
	
        % YOUR NAMES GO HERE
	{\bf\quad   10/22 \quad  第八周/星期二} % <---- Fill in your names here!
\end{center} 

\begin{enumerate}\setcounter{enumi}{69}
    \item 证明:连续n个整数中恰有一个被n整除.
 \item 
 \begin{itemize}
 \item[(1)] 证明:完全平方数模3同余于0或1,模4同余于0或1,模5同余于0,1或4;
 \item[(2)] 证明:完全立方数模9同余于0或±1;整数的四次幂模16同余于0或1.
 \end{itemize}
 \item 设$a$是奇数,n是正整数,证明:$a^{2^n}\equiv 1\ mod\ 2^{n+2}$.
 \item 设$m,n$都是正整数且有$m=nt$,则模n的任何一个同余类
 \begin{equation}
 \{x\in\mathbb{Z}|x\equiv r\ mod\ n\}\nonumber
 \end{equation}
 可表示为t个模m的(两两不同的)同余类
 \begin{equation}
 \{x\in\mathbb{Z}|x\equiv r+in\ mod\ m\}(i=0,1,...,t-1)\nonumber
 \end{equation}
 之并.
 \item 计算如下同余方程({\color{blue} 注:需要有计算过程}):\begin{enumerate}
     \item $5x\equiv 11 \mod 13$
     \item $29x \equiv 7  \mod 17$
     \item $26x \equiv 34 \mod 43$
 \end{enumerate}
 \item 设p为奇素数,证明:\\
    (1)\quad $\tbinom{p-1}{i-1} \equiv (-1)^{i-1} \mod p$;\\
    {\color{red} (选做)}(2)\quad $\sum\limits_{i=1}^{p-1} 2^i\cdot i^{p-2} \equiv \sum\limits_{i=1}^{\frac{p-1}{2}} i^{p-2} \mod p$。({\color{blue}提示:利用(1),以及证明(2)两边在模p意义下等于$-\frac{1}{p}(2^p-2)$})
\end{enumerate}

{\color{red} 注意:10.22和10.24作业应于10.24-10.29期间提交}
\newpage
\head

\begin{center} % Everything within the center environment is centered.
	{\Large \bf 第十二次作业} % <---- Don't forget to put in the right number
	\vspace{2mm}
	
        % YOUR NAMES GO HERE
	{\bf\quad   10/24 \quad  第八周/星期四} % <---- Fill in your names here!
\end{center} 

\begin{enumerate}\setcounter{enumi}{75}

\item 判定如下同余方程组是否有解,如果有解求出解集:
\begin{enumerate}
 \item  $\begin{cases}
 x\equiv 4321 \mod\ 440533,\\
 x\equiv 138344 \mod\ 563137,\\ 
 \end{cases}$
 \item  $\begin{cases}
 x\equiv 4321 \mod\ 266243,\\
 x\equiv 13834 \mod\ 478997.\\ 
 \end{cases}$
\end{enumerate}
{\color{red} 注: 这题的目的是让大家感受一下,当$m_i$比较大时,不同方法的效率. 课堂上有部分同学没有按照辗转相除法来求解同余方程组, 大家可以先试试用自己的方法解这一道题. 比较一下,辗转相除法和自己的方法那个更高效. 这一题允许大家用计算器做加减乘.}

\item 利用中国剩余定理求解下列同余方程组:
\[\begin{cases}
 2x\equiv \ 7 \mod\ 11,\\
 3x\equiv 12 \mod\ 17,\\
 5x\equiv \ 3 \mod\ 19.
 \end{cases}\]
 \item 求解下列同余方程组:
\[\begin{cases}
 x \equiv 11 \mod\ \ 40,\\
 x \equiv 31 \mod\ 100,\\
 x \equiv 45 \mod\ \ 98.
 \end{cases}\]

 \item (a). 设$p$为素数, $r$为正整数. 求 $\varphi(p^r)$. (其中$\varphi$为欧拉函数.) \\
 (b). 设$p,q$为不同的素数. 求$\varphi(pq)$.

 \item 证明:设p为素数,则有$(p-1)! \equiv -1 \mod p$(威尔逊定理). ({\color{blue} 提示: $1,\cdots,p-1$ 中除了$1$和$-1$外,其它元素可两两配对使得它们乘积模$p$同余于$1$.})
\item 设p为奇素数,如果$r_1,\cdots,r_{p-1}$与$r_1^{'},\cdots,r_{p-1}^{'}$都过模p的非零同余类$\{ [1],[2],\cdots,[p-1]\}$,证明:$r_1r_1^{'},\cdots,r_{p-1}r_{p-1}^{'}$不过模p的非零同余类$\{ [1],[2],\cdots,[p-1]\}$,即证明存在$i\neq j$,使得$r_ir_i^{'} \equiv r_jr_j^{'} \mod p$. ({\color{blue}提示: 威尔逊定理.})

{\color{red} 以下题目选做. 以后想学数论的同学必做.}
 
 \item 证明:对于任意正整数$n$,都存在$n$个连续正整数,使得它们其中每个数都不是素数的幂次(即不为$p^{\alpha}$,其中$p$为素数,$\alpha$为正整数).
 \item 设m为正整数,n为整数,证明:数2n可以表示为两个与m互素的整数之和({\color{blue} 提示:我们先证明一个引理:对于m为正整数,n为整数,存在整数$a,b$且满足$(a,m)=1,(b,m)=1$,使得$2n\equiv a+b \mod m$})

%\item (1) 设$m,n$为正整数,$(m,n)=1$,证明:$m^{\varphi(n)}+n^{\varphi(m)} \equiv 1 \mod mn$;\\
%(2) 设$a,m$为正整数,证明:$a^m\equiv a^{m-\varphi(m)} \mod m$.
\item 给定素数$p>5$,对于$k\in\{1,\cdots,p-1\}$,我们在模$q=p^n$意义下定义$\frac{1}{k}\equiv [x_k] \mod q,\text{其中}x_k\text{满足}x_k\cdot k \equiv 1 \mod q$,证明下列式子成立:\\
    (1)\quad$\sum \limits_{k=1}^{p-1} \frac{1}{k^4} \equiv 0 \mod p$;\\
    (2)\quad$\sum \limits_{k=1}^{p-1} \frac{1}{k^3} \equiv 0 \mod p^2$.
\end{enumerate}

{\color{red} 注意:10.22和10.24作业应于10.24-10.29期间提交}


\newpage
\head

\begin{center} % Everything within the center environment is centered.
	{\Large \bf 第十三次作业} % <---- Don't forget to put in the right number
	\vspace{2mm}
	
        % YOUR NAMES GO HERE
	{\bf\quad   10/29 \quad  第九周/星期二} % <---- Fill in your names here!
\end{center} 

\begin{enumerate}\setcounter{enumi}{84}

    \item 计算$\varphi(360),\varphi(429)$.
    
    \item (1)证明: 当$n\geq 3$时, $\varphi(n)$是偶数;\\
    (2)证明: 当$n\geq 2$ 时, 不超过n且与n互素的正整数之和是$\frac{n\varphi(n)}{2}$.
    
    \item (1) 求$3^{421}$十进制表示中的末两位数码.\\
          (2) 求$18^{1001}$十进制表示中的末两位数码.
    \item 设$\gcd(a,10)=1$,证明:$a^{20} \equiv 1 \mod 100$.

    
\item 设m,n为正整数,$\gcd(m,n)=1$.证明:$m^{\varphi(n)}+n^{\varphi(m)} \equiv 1 \mod mn$.

    \item 设$a$与$m$为正整数. 记群同态 $\mathbb Z/m\mathbb Z \rightarrow \mathbb Z/m\mathbb Z, \overline{x}\mapsto \overline{ax}$ 为$\varphi_a$. 证明: 
 对于任意$b\in \mathbb Z$, 若$\varphi_a^{-1}(\overline{b})\neq \emptyset$, 则 $|\varphi_a^{-1}(\overline{b})|=|\ker(\varphi_a)|=\gcd(a,m)$.




\item 设$q$为素数, $k$为域. 证明:\\
    (1) $\varphi\colon \mathbb Z\rightarrow k, n\mapsto n\cdot 1_k$ 为环同态. (注:此处$\cdot$ 不是$k$中乘法, 是取$1_k$的倍数.)\\
    (2) 若$\varphi$不是单同态,则理想$\ker(\varphi)$的正生成元为素数, 记为$p$. \\
    (3) 若$k$为有限域, 则$p\mid |k|$. (提示: $k$ 可写为形如 $\{a+n\cdot 1_k\mid n\in \mathbb Z\}$ 的子集的无交并, 这些子集的元素个数均为$p$.)\\
    (4) 若$k$为$q$元域, 则$k$与$\mathbb{F}_q:=\mathbb Z/q\mathbb Z$同构. (提示: 记 $n_k:=n\cdot 1_k$, 则 $k=\{0_k,1_k,\cdots, (q-1)_k\}$.)

\end{enumerate}

{\color{red} 注意:10.29和10.31作业应于10.31-11.5期间提交}

\newpage
\head

\begin{center} % Everything within the center environment is centered.
	{\Large \bf 第十四次作业} % <---- Don't forget to put in the right number
	\vspace{2mm}
	
        % YOUR NAMES GO HERE
	{\bf\quad   10/31 \quad  第九周/星期四} % <---- Fill in your names here!
\end{center} 

\begin{enumerate}\setcounter{enumi}{91}

    \item (习题6.1) 证明在群中\\
    (1) 元素$x$与它的逆的阶相同.\\
    (2) 元素$x$与它的共轭的阶相同.($x$与$y$在$G$中共轭$\iff \exists g\in G, s.t. g^{-1} x g=y$ )\\
    (3) 元素$xy$与$yx$的阶相同.\\
    (4) 元素$xyz$与$zyx$的阶不一定相同.

    \item (习题6.2) 证明$\frac{3}{5}+\frac{4}{5} i \in \mathbb{C}^{\times}$的阶为无穷.

    \item (习题6.3) 设
    $$ A=\begin{pmatrix}0&-1\\1&0\end{pmatrix}, B=\begin{pmatrix}0&1\\-1&-1\end{pmatrix}. $$
    试求$A$,$B$,$AB$和$BA$在$GL_2(\mathbb{R})$中的阶.

    \item (习题6.4) 证明群中元素$a$的阶$\leq 2$当且仅当$a=a^{-1}$.

    \item (习题6.5) 证明如果群$G$中任何元素的阶$\leq 2$,则$G$是阿贝尔群.

    \item (习题6.6) 设$x$在群中的阶是$n$,求$x^k(k\in \mathbb{Z})$的阶.

    \item {\color{red} (选做)} 设$p$是素数,试求$\mathbb{Z}/p\mathbb{Z} \times \mathbb{Z}/p\mathbb{Z}$ 有多少个$p$阶元?有多少个$p$阶子群?

    \item {\color{red} (选做)} 给出$\mathbb{Z}/m_1\mathbb{Z} \times \mathbb{Z}/m_2\mathbb{Z} \times \dots
    \times \mathbb{Z}/m_n\mathbb{Z}$为循环群的充要条件.

\end{enumerate}

{\color{red} 注意:10.29和10.31作业应于10.31-11.5期间提交}


\newpage
\head

\begin{center} % Everything within the center environment is centered.
	{\Large \bf 第十五次作业} % <---- Don't forget to put in the right number
	\vspace{2mm}
	
        % YOUR NAMES GO HERE
	{\bf\quad   11/5 \quad  第十周/星期二} % <---- Fill in your names here!
\end{center} 

\begin{enumerate}\setcounter{enumi}{99}

    \item (1) 记$G=\left\langle\begin{pmatrix} 1&1\\-1&0\\\end{pmatrix}\right\rangle\subset \mathrm{GL}_2(\mathbb Q)$. 求群$G$所有子群.\\
          (2) 记$G=\left\langle\begin{pmatrix} 1&1\\0&1\\\end{pmatrix}\right\rangle\subset \mathrm{GL}_2(\mathbb Q)$. 求群$G$所有子群.

    \item (习题6.10) 设 $p$ 为奇素数, $X$ 为2阶整系数矩阵, 而 $I=\begin{pmatrix}
        1 & 0\\
        0 & 1
    \end{pmatrix}$. 如果 $I+pX\in \text{SL}_2(\mathbb{Z})$ 的阶有限, 证明 $X=0$.

    \item (习题6.12) 设 $G=\langle g\rangle$ 为 $n$ 阶循环群. 证明: 元素 $g^k$ 与 $g^l$ 有相同的阶当且仅当 $\gcd(k,n)=\gcd(l,n)$. 
    \item (习题6.13) 设 $G=\langle g \rangle$ 为100阶循环群. 试求\\
    (1)\quad 所有满足 $a^{20}=1$ 的元素 $a$. \\
    (2)\quad 所有阶为20的元素 $a$. 
    \item (习题6.21) $S^1=(\{z\in \mathbb{C}\mid |z|=1\},\cdot)$ 的任意有限子群均为循环群.

    \item 设 $G_1$ 和 $G_2$ 为两群. 设$\varphi_i\colon G_i\rightarrow G_i$为$G_i$的自同构($i=1,2$).  证明
    \[\varphi_1\times\varphi_2 \colon G_1\times G_2 \rightarrow G_1\times G_2, \quad (g_1,g_2)\mapsto (\varphi_1(g_1),\varphi_2(g_2))\]
    为群$G_1\times G_2$的自同构, 且 
    \begin{equation*} 
        \psi\colon \mathrm{Aut}(G_1)\times\mathrm{Aut}(G_2) \rightarrow \mathrm{Aut}(G_1\times G_2), \quad (\varphi_1,\varphi_2)\mapsto \varphi_1\times\varphi_2 \qquad \qquad (*)
    \end{equation*}
    为群的单同态.  
    \item 设$p,q$为两不同的素数. 令 $G_1=\mathbb{Z}/p\mathbb{Z}$, $G_2=\mathbb{Z}/q\mathbb{Z}$. \\
    (1)\quad 求 $G=G_1\times G_2$ 所有生成元.\\
    (2)\quad 写出 $G$ 的所有子群.\\
    (3)\quad 证明此时, $(*)$ 中定义的$\psi$为同构.
   

    \item {\color{red} (选做)} 设$p$为素数. 令 $G_1=G_2=\mathbb{Z}/p\mathbb{Z}$. \\
    (1)\quad 写出 $G=G_1\times G_2$ 的所有子群.\\
    (2)\quad 证明此时, $(*)$ 中定义的$\psi$为不是同构.

    \item {\color{red} (选做)} (习题6.9) 设 $m$ 是奇正整数且不是素数幂次.\\
    (1)\quad 求 $(\mathbb{Z}/m\mathbb{Z})^\times$ 中2阶元的个数.\\
    (2)\quad 证明
    $$ \prod_{g\in(\mathbb{Z}/m\mathbb{Z})^\times} g=1$$
    ({\color{blue}提示: 习题6.7,6.8})
    
\end{enumerate}

{\color{red} 注意:11.5和11.7作业应于11.7-11.12期间提交}


\newpage
\head

\begin{center} % Everything within the center environment is centered.
	{\Large \bf 第十六次作业} % <---- Don't forget to put in the right number
	\vspace{2mm}
	
        % YOUR NAMES GO HERE
	{\bf\quad   11/7 \quad  第十周/星期四} % <---- Fill in your names here!
\end{center} 

\begin{enumerate}\setcounter{enumi}{108}

    \item 已知 $(\mathbb{Z}/17\mathbb{Z})^*$ 为循环群, $\overline{3}$ 为其生成元: \\
    (1) 写出对数表:\\
    \begin{tabular}{|c|c|c|c|c|c|c|c|c|c|c|c|c|c|c|c|c|}
    \hline
    $k$ & 0 & 1 & 2 & 3 & 4 & 5 & 6 & 7 & 8 & 9 & 10 & 11 & 12 & 13 & 14 & 15\\ \hline
    $3^k$ & $\overline{1}$ & $\overline{3}$ & & & & & & & & & & & & & & \\ \hline
    \end{tabular}\\
    (2) 利用对数表求解同余方程: $10^x\equiv5(\text{mod}\ 17)$.\\
    (3) 利用对数表求解同余方程: $x^6\equiv2(\text{mod}\ 17)$.

    \item 利用中国剩余定理将 $$(\mathbb{Z}/(2^3\times 5\times 7)\mathbb{Z})\times (\mathbb{Z}/(2\times 7^2)\mathbb{Z})\times (\mathbb{Z}/(5\times 11)\mathbb{Z})$$ 化为标准形式.  \\

    {\color{red} 以下题目选做. 按如下的步骤证明有限交换群的结构定理}

    \item 设 $G$ 为有限交换群. $p$ 为素数.\\
    (1)\quad $G(p):=\{g\in G\mid \exists k\in\mathbb{N},\text{s.t. } g^{p^k}=1_{G}\}$ 为 $G$ 的子群.\\
    (2)\quad 集合 $\{\text{素数  }\ p\mid \exists g\in G,\text{s.t. }p\mid \text{ord}(g)\}$ 为有限集. 记为 $\{p_1,p_2,\cdots,p_s\}$. \\
    (3)\quad 映射 
    \begin{align*}
    \varphi: G(p_1)\times G(p_2)\times \cdots\times G(p_s)&\longrightarrow G\\
    (g_1,g_2,\cdots,g_s)&\longmapsto g_1g_2\cdots g_s
    \end{align*}
    为群同态.\\
    (4)\quad $\varphi$ 为单同态{(提示: 设 $\varphi(g_1,\cdots,g_s)=1$ , 其中 $g_i^{p_i^{\alpha_i}}=1_G$, 取 $M_i$ 满足 $M_i\equiv 1(\text{mod}\ p_i^{\alpha_i}),M_i\equiv0 (\text{mod}\  p_j^{\alpha_j})(\forall j\not=i)$,则 $1_G=\varphi((g_1\cdots g_s)^{M_i}=g_i$)}\\
    (5)\quad $\varphi$ 为满同态. (提示: 设 $g\in G$, $n$= ord($g$)= $p_1^{\alpha_1}\cdots p_s^{\alpha_s}(\alpha_i \geq 0)$, $n_i:=n/p_i^{\alpha_i}$, 则 gcd($n_1,\cdots,n_s$)=1$\implies\exists x_1,\cdots,x_s,\text{s.t. } \sum_{i} n_ix_i=1,g=\prod_i (g^{n_i})^{x_i}$)

    \item 设 $G$ 为有限交换群, $p$ 为素数,若 $G=G(p)$, 设 $g_0$ 为 $G$ 中一个阶数最大的元素, 即 $p^\alpha=\text{ord}(g_0)=\max_{g\in G} \text{ord}(g)$ . 则存在 $H_0\leq G$ 使得 映射 
    \begin{align*}
    \varphi: \langle g_0\rangle \times H_0&\longrightarrow G\\
    (g_0^i,h)&\longmapsto g_0^ih
    \end{align*}
    为群同构. 请按如下步骤完成证明:\\
    取$H_0$ 为 $\Sigma:=\{H\leq G\mid \langle g_0\rangle \cap H=\{1_G\}\}$ 中阶数最大的一个子群, 并如上面构造映射 $\varphi$.  \\
    (1)\quad 验证 $\varphi$ 为群的单同态.\\
    (2)\quad $\forall g\in G,g^{p^\alpha}=1_G$.\\
    (3)\quad 若 $g^p\in \text{Im}\ \varphi$, 则存在 $i\in\mathbb{Z}$, s.t. $(gg_0^i)^p\in H_0$. (提示: 若$g^p=g_0^kh_0$,则 $p\mid k$ ,$(gg_0^{-\frac{k}{p}})^p=h_0\in H$)\\
    (4)\quad 若 $((gg_0)^i)^p \in H_0$, 则 $g\in \text{Im}\varphi$.(提示: 否则 $\langle gg_0^i\rangle \cdot H_0\notin \Sigma\implies \exists h\in H_0,j,l,\text{s.t. } g_0^j=(gg_0^i)^l\cdot h\not=1_G\implies p\nmid l\implies g=g_0^{-i} (g_0^j h^{-1})^{l'}\in \text{Im}\ \varphi$, 其中 $ll'\equiv 1$(mod $p^\alpha$) )\\
    (5)\quad 结合(2)(3)(4),说明 $\varphi$ 为满射. 
\end{enumerate}

{\color{red} 注意:11.5和11.7作业应于11.7-11.12期间提交}


\newpage
\head

\begin{center} % Everything within the center environment is centered.
	{\Large \bf 第十七次作业} % <---- Don't forget to put in the right number
	\vspace{2mm}
	
        % YOUR NAMES GO HERE
	{\bf\quad   11/19 \quad  第十二周/星期二} % <---- Fill in your names here!
\end{center} 

\begin{enumerate}\setcounter{enumi}{112}
    \item 证明阶$\leq5$ 的群都是阿贝尔群.
    \item 在同构意义下确定所有的4阶群.
    \item 设$g_1,g_2$是群 G 的元素,$H_1,H_2$是 G 的子群,证明下列两条等价:
    \begin{enumerate}
\item[1)]$g_1H_1\subseteq g_2H_2;$
\item[2)]$H_1\subseteq H_2$且$g_2^{-1}g_1\in H_2$.
\end{enumerate}

\item 设$g_1,g_2$是群 G 的元素,$H_1,H_2$是 G 的子群。证明如果$g_1H_1\cap g_2H_2\neq\varnothing$,则它
是关于子群$H_1\cap H_2$的左陪集.
\item 如果$H$与$K$是$G$的子群且阶互素,证明$H\cap K=\{1\}$.
\item 设$G=\bigsqcup_{i\in I} a_{i}H$,对每个$i$,取$s_{i}\in a_{i}H$.证明:$S=\{s_{i}|i\in I\}$为左陪集代表元系,即$G=\bigsqcup_{i\in I} s_{i}H$.
\item 若$aH=Hb$,则$aH=Ha=bH=Hb$.
\item {\color{red} (选做)} $A=\mathbb{Z}/m_{1}\mathbb{Z}\times\cdots\times \mathbb{Z}/m_{n}\mathbb{Z}$,其中$2\leq m_{1}\mid m_{2}\cdots\mid m_{n},\\
A[d]:=\left\{a\in A|da=0\right\}$,证明:
\begin{enumerate}
    \item[(1)] $A[d]$为$A$的子群;
    \item[(2)] $\# A[d]=\prod_{i=1}^{n}gcd(d,m_{i})$;
    \item[(3)] 若$\mathbb{Z}/m_{1}\mathbb{Z}\times\cdots\times \mathbb{Z}/m_{n}\mathbb{Z}\cong \mathbb{Z}/m_{1}^{'} \mathbb{Z}\times\cdots\times \mathbb{Z}/m_{n^{'}}^{'} \mathbb{Z}$,其中$2\leq m_{1}\mid m_{2}\cdots\mid m_{n},2\leq m_{1}^{'}\mid m_{2}^{‘}\cdots\mid m_{n^{'}}^{'}$.则$n=n^{'}$且$m_i=m_i^{'}(\forall i=1,\cdots,n)$.\\
    (\color{blue}提示: $A\cong A^{'}\Rightarrow \# A[d]=\# A^{'}[d](\forall d)$)
\end{enumerate}

\end{enumerate}

{\color{red} 注意:11.19和11.21作业应于11.21-11.26期间提交} 



\newpage
\head

\begin{center} % Everything within the center environment is centered.
	{\Large \bf 第十八次作业} % <---- Don't forget to put in the right number
	\vspace{2mm}
	
        % YOUR NAMES GO HERE
	{\bf\quad   11/21 \quad  第十二周/星期四} % <---- Fill in your names here!
\end{center} 

\begin{enumerate}\setcounter{enumi}{120}
    \item 设 $\varphi:G\to G^{\prime}$ 为群同态:,若 $N^{\prime}\lhd G^{\prime}$,则 $\varphi^{-1}(N^{\prime})\lhd G$.
    \item 设 $H\leq G,N\lhd G$,则 $HN:=\{hn|h\in H,n\in N\}$为G的子群.
    \item 请给出 $X=\{A,B,C\}$上的所有等价关系.
    \item 请证明等价关系中的3条公理相互独立,即
    \begin{enumerate}
        \item[1)] 存在关系满足自反性、对称性,但不满足传递性;
        \item[2)] 存在关系满足自反性、传递性,但不满足对称性;
        \item[3)] 存在关系满足对称性、传递性,但不满足自反性.
    \end{enumerate}
    \item \begin{enumerate}
        \item[(1)] 设$f:X\to Y$为集合之间的映射,则 $\mathcal{R}_f:=\{(x_1,x_2)|f(x_1)=f(x_2)\}$为$X$上的等价关系.
        \item[(2)] 若$\mathcal{R}$为$X$上等价关系,则存在映射 $g:X\to Y$使得 $\mathcal{R}=\mathcal{R}_g$.
    \end{enumerate}
    \item {\color{red} (选做)} 若$H\lhd G$,则
    \begin{enumerate}
        \item[(a)] $(G/H,\cdot)$构成群;
        \item[(b)] $\varphi:G\to G/H,g\mapsto gH$为群的满同态;
        \item[(c)] $\ker \varphi =H$.
    \end{enumerate}
    \item {\color{red} (选做)} 若$\varphi:G\to G^{\prime}$ 为群同态,则im $\varphi \cong G/\ker \varphi$.
    \item {\color{red} (选做)} 设$R$为环,$I\lhd R$为理想,则
    \begin{enumerate}
        \item[(a)] $(R/I,+,\cdot)$构成环;
        \item[(b)] $\varphi:R\to R/I,r\mapsto r+I$为环的满同态;
        \item[(c)] $\ker \varphi =I$.
    \end{enumerate}
    \item {\color{red} (选做)} 若$\varphi:R\to R^{\prime}$ 为环同态,则im $\varphi \cong R/\ker \varphi$.
    \item {\color{red} (选做)}设$R$为整环,在
    $$R\times R\backslash\{0\}=\{(p,q)|p\in R,q\in R\backslash\{0\}\}$$
    上定义关系
    $$(p,q)\sim(s.t)\overset{def}{\Leftrightarrow}pt=sq$$
    \begin{enumerate}
        \item[(a)] 证明"$\sim$"为$R\times R\backslash\{0\}$上的等价关系.
        \item[(b)] 记$$\frac{p}{q}:=\{(s,t)|(s,t)\sim(p,q)\}$$
        $$Frac(R):=\{\frac{p}{q}|p\in R,q\in R\backslash\{0\}\}$$
        证明:$\begin{cases}\frac{p}{q}+\frac{s}{t}:=\frac{pt+sq}{qt}\\\frac{p}{q}\cdot\frac{s}{t}:=\frac{ps}{qt}\end{cases}$是良定义的.
        \item[(c)] 证明:$(Frac(R),+,\cdot)$构成域.
        \item[(d)] 证明:$\varphi:R\to Frac(R),r\mapsto \frac r1$为环的单同态.
    \end{enumerate}
\end{enumerate}

{\color{red} 注意:11.19和11.21作业应于11.21-11.26期间提交} 

\newpage
\head

\begin{center} % Everything within the center environment is centered.
	{\Large \bf 第十九次作业} % <---- Don't forget to put in the right number
	\vspace{2mm}
	
        % YOUR NAMES GO HERE
	{\bf\quad   11/26 \quad  第十三周/星期二} % <---- Fill in your names here!
\end{center} 

\paragraph{定义}若$(\mathbb{Z}/m\mathbb{Z})^\times$循环,且$g\ mod\ m$生成$(\mathbb{Z}/m\mathbb{Z})^\times$,则称$g$为模$m$的一个原根.

\begin{enumerate}\setcounter{enumi}{130}
 \item 设$p$是奇素数.证明:模$p$的任意两个原根之积不是模$p$的原根.
 \item 设$p$是奇素数,对于任意的$0\leq i \leq p-2$,证明都有$\sum\limits_{x=1}^{p} x^i \equiv 0 \mod p$成立
 \item 设$n,a$都是正整数且$a>1$,试求$a$在群$(\mathbb{Z}/(a^n-1)\mathbb{Z})^\times$的阶,并证明:$n\mid \varphi(a^n-1)$.
 \item 设$m$是正整数.整数$a$和$b$对于模$m$的阶分别是$s$及$t$,且$(s,t)=1$.证明:$ab$模$m$的阶是$st$.
 \item \begin{itemize}
 \item[(1)]对$p=3,5,7,11,13,17,19,23$,求模$p$的最小正原根(直接给出答案即可);
 \item[(2)]求模11的所有原根(需要计算过程)
 \end{itemize}
 \item 设$\varphi :G\twoheadrightarrow H$为群的满同态.证明:若$G$为循环群,则$H$也为循环群.
 \item 设$G$是一个n阶有限群,若对任一$n$的正因子$m$,$G$中至多只有一个$m$阶子群,证明$G$是循环群
 \item ({\color{red} 选做}) 设$G$为有限阿贝尔群,取正整数$d$满足$d||G|$,证明$G$中有一个$d$阶子群
\end{enumerate}

{\color{red} 注意:11.26和11.28作业应于11.28-12.3期间提交}

\newpage
\head

\begin{center} % Everything within the center environment is centered.
	{\Large \bf 第二十次作业} % <---- Don't forget to put in the right number
	\vspace{2mm}
	
        % YOUR NAMES GO HERE
	{\bf\quad   11/28 \quad  第十三周/星期四} % <---- Fill in your names here!
\end{center} 


\begin{enumerate}\setcounter{enumi}{138}
 \item \begin{itemize}
     \item[(1)] 求模$11^{101}$的一个原根(要求计算过程)
     \item[(2)] 求模$18$的所有原根(要求计算过程)
 \end{itemize}
 \item 设$p$是奇素数,假设存在数$a,p\nmid a$,使得对$p-1$的所有素因子$q$,有$a^{(p-1)/q}\not\equiv 1\ (mod\ p)$,证明:$a$是模$p$的原根.
 \item 设$p$与$q=2p+1$都是素数时.证明
 \begin{itemize}
 \item[(1)] 当$p\equiv1\ (mod\ 4)$时,$2$是模$q$的原根;
 \item[(2)] 当$p\equiv3\ (mod\ 4)$时,$-2$是模$q$的原根.
 \end{itemize}

 \item 给定义奇素数$p$,求所有$\mathbb{Z}\rightarrow \mathbb{Z}$的函数$f$,满足对任意的整数$m,n$都有$f(mn)=f(m)f(n)$,以及如果$m\equiv n \mod p$,则有$f(m)=f(n)$.

 \item 设$n>1$是正整数,证明下述命题等价:\begin{itemize}
     \item[(1)] 对任意的非零自然数$a$,都有$n|(a^n-a)$
     \item[(2)] 对$n$的任一素因子$p$,都有$p$恰好整除$n$且$(p-1)|(n-1)$ 
 \end{itemize}

 \item({\color{red}选做}) 证明:群$G$是循环群当且仅当$G$的任一子群都形如$G^m=\{g^m|g\in G\}$,其中$m$是非负整数。\\(提示:分$G$中有无限阶元和仅有限阶元的情况讨论,并且可以知道在后者情况下$G$的所有元素的阶构成的集合是有限集)
\end{enumerate}

{\color{red} 注意:11.26和11.28作业应于11.28-12.3期间提交}


\newpage
\head

\begin{center} % Everything within the center environment is centered.
	{\Large \bf 第二十一次作业} % <---- Don't forget to put in the right number
	\vspace{2mm}
	
        % YOUR NAMES GO HERE
	{\bf\quad   12/3 \quad  第十四周/星期二} % <---- Fill in your names here!
\end{center} 

\begin{enumerate}\setcounter{enumi}{144}
 \item 计算$(\frac{17}{23}),(\frac{19}{37}),(\frac{60}{79}),(\frac{92}{101})$.

 \item \begin{enumerate}
    \item[(1)] 确定以$-3$为二次剩余的素数;
    \item[(2)] 确定以$5$为二次剩余的素数.
 \end{enumerate}

 \item 设$p=4k+1$是素数,$a$是$k$的因子,证明$(\frac{a}{p})=1$.

 \item 设$p$是素数,$p\equiv1\ (mod\ 4)$,证明:
 \begin{itemize}
 \item[(1)] $\sum\limits_{\substack{r=1\\(\frac{r}{p})=1}}^{p-1}r=\frac{p(p-1)}{4}$;
 \item[(2)] $\sum\limits_{a=1}^{p-1}a(\frac{a}{p})=0$;
 \item[(3)] $\sum\limits_{k=1}^{\frac{p-1}{2}}[\frac{k^2}{p}]=\frac{(p-1)(p-5)}{24}$.
 \end{itemize}
 (提示:$(\frac{p-r}{p})=(\frac{-1}{p})(\frac{r}{p})$,然后利用带余除法$k^2=[\frac{k^2}{p}]p+r_k$)
 
 \item 设$p$是素数,$p\equiv 3\ (mod\ 4)$,且$p>3$,证明:
 \begin{itemize}
 \item[(1)] $\sum\limits_{\substack{r=1\\(\frac{r}{p})=1}}^{p-1}r\equiv0\ (mod\ p)$;
 \item[(2)] $\sum\limits_{a=1}^{p-1}a(\frac{a}{p})\equiv0\ (mod\ p)$.
 \end{itemize}
 (提示:注意到$\sum\limits_{\substack{r=1\\(\frac{r}{p})=1}}^{p-1}r\equiv\sum\limits_{k=1}^{\frac{p-1}{2}}k^2\ (mod\ p)$)

 \item 设$n>1$,$p=2^n+1$是素数。证明:模$p$的原根集合与模$p$的二次非剩余集合相同;进而证明$3,7$都是模$p$的原根.

 \newpage
 \item 设$p$是奇素数,$a$是整数.
 \begin{itemize}
 \item[(1)] 证明:同余方程$x^2-y^2\equiv a\ (mod\ p)$必有解;
 \item[(2)] 若$(x,y)$和$(x',y')$均是上述同余方程的解,当$x\equiv x'$且$y\equiv y'\ (mod\ p)$时,我们将$(x,y)$和$(x',y')$看成模$p$的同一个解.证明:(1)中同余方程的解数是$p-1$(如果 $p\nmid a$)或$2p-1$(如果 $p\mid a$).
 \end{itemize}
 (提示:考虑集合$A=\{k^2\}\subset \mathbb{F}_p$与集合$B=\{k^2+a\}\subset \mathbb{F}_p$;第二问考虑分解$x^2-y^2=(x+y)(x-y)$并利用原根)


\end{enumerate}

{\color{red} 注意:12.3和12.5作业应于12.5-12.10期间提交}

\newpage
\head

\begin{center} % Everything within the center environment is centered.
	{\Large \bf 第二十二次作业} % <---- Don't forget to put in the right number
	\vspace{2mm}
	
        % YOUR NAMES GO HERE
	{\bf\quad   12/5 \quad  第十四周/星期四} % <---- Fill in your names here!
\end{center} 

\begin{enumerate}\setcounter{enumi}{151}

\item 求所有的素数$p$使得$x^2-15$在$\mathbb{F}_p[x]$中可约.

 \item 设$a$是奇数,则有:\begin{enumerate}
    \item[(1)] $x^2\equiv a \mod 2$对所有$a$都有解;
    \item[(2)] $x^2\equiv a \mod 4$有解的充要条件是$a\equiv 1 \mod 4$,并且在此条件满足时有两个不同的解;
    \item[(3)] $x^2\equiv a \mod 2^k(k\geq 3)$有解的充要条件是$a\equiv 1 \mod 8$,并且在此条件成立时恰有四个解:如果$x_0$是一个解,则$\pm x_0,\pm x_0+2^{k-1}$是所有解.
\end{enumerate}

\item 设$p$是奇素数,证明:$\mathbb{F}_p[x]$中形如$x^2+\alpha x+\beta$ 的二次多项式中,共有$\frac{p(p-1)}{2}$个不可约多项式.
 (提示:$x^2+\alpha x+\beta=(x+2^{-1}\alpha)^2+\beta-4^{-1}\alpha^2$,对$(\frac{\beta}{p})$
 进行分类讨论并运用151题结论)
 
 \item 设$p$是奇素数,$f(x)=ax^2+bx+c$且$p\nmid a$.记\begin{equation}
 D=b^2-4ac.\nonumber
 \end{equation}证明
 \begin{equation}\nonumber
 \sum_{x=0}^{p-1}(\frac{f(x)}{p})=\left\{
 \begin{aligned}
 -(\frac{a}{p}),如果p\nmid D,\\
 (p-1)(\frac{a}{p}),如果p\mid D.
 \end{aligned}
 \right.
 \end{equation}

 \item 设$\mathbb{F}$是域,$a\in \mathbb{F}$,在多项式环$\mathbb{F}[x]$上证明:\begin{enumerate}
    \item[(1)] 若$n$是正整数,则$x-a \mid x^n-a^n$;
    \item[(2)] 若$n$是正奇数,则$x+a \mid x^n+a^n$.
\end{enumerate}

\end{enumerate}

{\color{red} 注意:12.3和12.5作业应于12.5-12.10期间提交}


\newpage
\head

\begin{center} % Everything within the center environment is centered.
	{\Large \bf 第二十三次作业} % <---- Don't forget to put in the right number
	\vspace{2mm}
	
        % YOUR NAMES GO HERE
	{\bf\quad   12/10 \quad  第十五周/星期二} % <---- Fill in your names here!
\end{center} 

\begin{enumerate}\setcounter{enumi}{156}

\item 对于下面的情形, 用欧几里得算法求 $(f(x),g(x))$: \begin{enumerate}
    \item[(1)] $F=\mathbb{Q},f(x)=x^3+x-1,g(x)=x^2+1$;
    \item[(2)] $F=\mathbb{F}_2,f(x)=x^7+\overline{1},g(x)=x^6+x^5+x^4+\overline{1}$;
    \item[(3)] $F=\mathbb{F}_3,f(x)=x^8+\overline{2}x^5+x^3+\overline{1},g(x)=\overline{2}x^6+x^5+\overline{2}x^3+\overline{2}x^2+\overline{2}$.
\end{enumerate}

\item 设 $m,n$ 是正整数, 证明: $F[x]$ 上多项式 $x^m-1$ 和 $x^n-1$ 的最大公因数是 $x^{(m,n)}-1$. 

\item 设 $f(x),g(x)\in F[x]$, 且 $f(x)$ 与 $g(x)$ 互素, 则对任意正整数 $n$, $f(x^n)$ 与 $g(x^n)$ 也互素.

\item (1) 求有理系数多项式 $\alpha(x),\beta(x)$ 使 $x^3\alpha(x)+(1-x)^2\beta(x)=1$;
    \par(2) 更一般地, 对于正整数 $m,n$, 求有理系数多项式 $u(x),v(x)$ 使 $x^mu(x)+(1-x)^nv(x)=1$.

\item 设 $f$ 和 $g$ 都是 $F[x]$ 中次数至少为1的多项式, 且不存在 $u\in F$, 使得 $f=ug$. 设 $d(x)$ 是 $u(x)$ 与 $V(x)$ 的最大公因子. 证明: 
    \begin{enumerate}
        \item[(1)] 存在多项式 $u(x),v(x)$, 使得 $\text{deg}\ u(x)<\text{deg}\ g(x)-\text{deg}\ d(x)$ 且 $d(x)=f(x)u(x)+g(x)v(x)$;
        \item[(2)] 此时 $\text{deg}\ v(x)<\text{deg}\ f(x)-\text{deg}\ d(x)$;
        \item[(3)] 符合 $(a)$ 中条件的多项式 $u(x),v(x)$ 是唯一确定的.
    \end{enumerate}

\item 设 $f(x),g(x)\in F[x]$ 满足 $g(x)\not=0$. 则 $\frac{f(x)}{g(x)}\in F(x)$, 其中 $F(x)$ 为 $F$ 上有理函数域. 下面对于形式分式的计算都是在分式域上进行. 则: 
    \begin{enumerate}
        \item[(1)] 设 $g(x)=a(x)b(x)$, 其中 $a(x)$ 与 $b(x)$ 互素且均非常数; 假设 deg$\ f<\text{deg}\ g$, 则存在唯一确定的 $r(x),s(x)\in F[x]$,  deg$\ r<\text{deg}\ a$, deg$\ s<\text{deg}\ b$, 使得 $\frac{f(x)}{g(x)}=\frac{r(x)}{a(x)}+\frac{s(x)}{b(x)}$;
        \item[(2)] 设 \(g(x)\) 为首项系数为1, 其标准分解是 \(g(x)=\prod\limits_{i=1}^{l}p_{i}^{m_{i}}(x)\).  假设 \(\deg f<\deg g\), 则存在唯一确定的多项式 \(h_{i}(x)\in F[x],\deg h_{i}<m_{i}\deg p_{i}(1\leq i\leq l),\) 使得

\(
\frac{f(x)}{g(x)}=\frac{h_{1}(x)}{p_{1}^{m_{1}}(x)}+\cdots+\frac{h_{l}(x)}{p_{l }^{m_{l}}(x)};
\)

\item[(3)] 设 \(p(x)\in F[x]\) 是不可约多项式, m是正整数. 则对于任意 \(h(x)\in F[x],\) 若

\(h(x)\neq 0\) 且 \(\deg h<m\deg p,\) 则存在唯一确定的多项式 \(\alpha_{i}(x)\in F[x](1\leq i\leq m)\) 使得 \(
\frac{h(x)}{p^{m}(x)}=\frac{\alpha_{m}(x)}{p(x)}+\cdots+\frac{\alpha_{1}(x)}{p^ {m}(x)},
\) 其中 \(\deg\alpha_{i}<\deg p;\)

\item[(4)] 证明: 每一个分子的次数小于分母的次数, 且分母有标准分解 \(f(x)=p_{1}^{m_{1}}(x)\cdots p_{l}^{m_{l}}(x)\) 的有理分式 \(\frac{g(x)}{f(x)}\) 是部分分式的和, 每个部分分式的分母是 \(p_{i}^{k_{i}}(x)(k_{i}=1,\cdots,m_{i};i=1,\cdots,l),\)  而分子次数小于 \(\deg p_{i}\).
    \end{enumerate}

{\color{red} 注意:12.10和12.12作业应于12.12-12.17期间提交}


\end{enumerate}
\newpage
\head

\begin{center} % Everything within the center environment is centered.
	{\Large \bf 第二十四次作业} % <---- Don't forget to put in the right number
	\vspace{2mm}
	
        % YOUR NAMES GO HERE
	{\bf\quad   12/12 \quad  第十五周/星期四} % <---- Fill in your names here!
\end{center} 

\begin{enumerate}\setcounter{enumi}{162}

\item 确定 \( \mathbb{F}_2[x] \) 与 \( \mathbb{F}_3[x] \) 中所有 2 次及 3 次的首项系数为 1 的不可约多项式.

\item 设直线 \( y = ax + b \) 交曲线 \( y^2 = x^3 + cx + d \) 于两点 \( (x_1, y_1), (x_2, y_2) \). 试用 \( x_1, y_1, x_2, y_2 \) 表示 \( a, b, c \) 和 \( d \). 


\item 设 \( f(x) \in \mathbb{F}_p[x], \deg f = p - 2 \). 若对所有 \( \alpha \in \mathbb{F}_p (\alpha \neq 0) \) 有 \( f(\alpha) = \alpha^{-1} \), 试确定 \( f(x) \). 

\item 
令分圆多项式 \( \Phi_n(x) = \prod_{k=1, (k,n)=1}^{n} (x - \zeta_n^k) \). 证明:\\
(1) \( \prod_{1 \leq d | n} \Phi_d(x) = x^n - 1. \)\\
(2) 如果 \( n \) 为大于 1 的奇数, 则 \( \Phi_{2n}(x) = \Phi_n(-x) \). \\
(3) \( \Phi_n(x) = \prod_{1 \leq d | n} (x^d - 1)^{\mu(\frac{n}{d})}, \) 其中 \( \mu \) 为莫比乌斯函数. \\
(注: 关于莫比乌斯函数的定义参考习题67和68, 允许不加证明地使用这两题中的结论)

\item 设 \( F \) 为 \( F' \) 的子域, \( f(x), g(x) \in F[x] \). 证明

1) $f$ 在 \( F[x] \) 中整除 $g$ 当且仅当 $f$ 在 \( F'[x] \) 中整除 $g$;

2) $f$ 与 $g$ 在 \( F[x] \) 中互素当且仅当 $f$ 与 $g$ 在 \( F'[x] \) 中互素.\\

{\color{red} 以下题目选做.}

\item 设 $p$ 为素数, $n$ 为正整数, \( F \) 为 \( p^n \) 元域.
\par (1) 证明:  \( F^{\times} \) 为 \( p^n-1 \) 循环群. (提示: 与 $n=1$ 时相似)
    \par (2) $d$ 为正整数, \( d|n \), 则 \( E := \{\alpha \in F | \alpha^{p^d} = \alpha\} \)  构成 $F$ 的子域. (提示: 在 \( F \) 上 \( (\alpha+\beta)^p = \alpha^p+\beta^p \))

\item 设 \( f \) 为 \( \mathbb{F}_p[x] \) 中 \( d \) 次首一不可约多项式, 则 $f|x^{p^n}-x\iff d|n$.

(提示: (右推左) 记 \( F' = \mathbb{F}_p[x]/f\mathbb{F}_p[x] \implies \overline{x} \in F' \) 为 \(f(x) \in F'[x] \) 的根 $\implies f$ 与 \( x^{p^d}-x \) 在 \( F'[x] \) 中不互素\\

(左推右) $ d' := \text{gcd}(n, d) \implies f|\text{gcd}(x^{p^n}-x, x^{p^d}-x) = x^{p^{d'}}-x \implies \overline{x}\in E:=\{\alpha\in F'|\alpha^{p^{d'}}\}\implies  F' \subset E \implies p^d\leq p^{d'}$ )

{\color{red} 注意:12.10和12.12作业应于12.12-12.17期间提交}

\end{enumerate}

\newpage
\head

\begin{center} % Everything within the center environment is centered.
	{\Large \bf 第二十五次作业} % <---- Don't forget to put in the right number
	\vspace{2mm}
	
        % YOUR NAMES GO HERE
	{\bf\quad   12/17 \quad  第十六周/星期二} % <---- Fill in your names here!
\end{center} 

\begin{enumerate}\setcounter{enumi}{169}

\item (习题5.8)设$f(x)$是实系数多项式,$a\in \mathbb{R}$,试决定$a$在下述多项式的零点重数:
\begin{enumerate}
    \item $f(x)-f(a)-f^{'}(a)(x-a)-\frac{f^{''}(a)}2(x-a)^2$;
    \item $f(x)-f(a)-\frac{x-a}2(f^{'}(x)+f^{'}(a))$.
\end{enumerate}

\item (习题5.9)证明多项式$f(x)=\sum^n_{k=0}\frac{x^k}{k!}$无重根.

\item (习题5.10)证明1是多项式$x^{2n}-nx^{n+1}+nx^{n-1}-1$的3重零点,其中$n\ge 2$.

\item (习题5.11)设$f(x)\in\mathbb{Q}[x]$在$\mathbb{Q}$上不可约,证明它一定没有多重的复根.

\item 请在$\mathbb{R}[x]$中分解多项式$x^5+1$和$x^5-2$.

\item 设$f(x)\in \mathbb{Z}[x]$,且$f(0)\equiv f(1)\equiv 1\;(mod \;2)$,证明:$f(x)$没有整数根.

\item ({\color{red}选做}) 设$\mathbb{F}=\mathbb{F}_p,n\ge 1$,
\begin{enumerate}
    \item 证明$$x^{p^{n}}-x=\prod_{P(x):\text{首一不可约},\deg P\mid n}P(x)$$
    \item 证明在$\mathbb{F}[x]$中存在$n$次不可约多项式.\\({\color{blue}提示:即证明$n$次不可约多项式个数大于0})
\end{enumerate}

\end{enumerate}
{\color{red} 注意:12.17和12.19作业应于12.19-12.24期间提交}

 \newpage
\head
 \begin{center}
 {\Large \bf 第二十六次作业}
 \vspace{2mm}
 
 {\bf\quad 12/19\quad 第十六周/星期四}
 \end{center}
 
 \begin{enumerate}\setcounter{enumi}{176}
 \item 在相应的环中判定不可约性
 \begin{itemize}
 \item [1)]$2x^3+3x+1\in\mathbb{Z}[x]$
 \item [2)]$2x^5+30x+90\in\mathbb{Q}[x]$
 \item [3)]$x^4-x^3-3x^2+8x+1\in\mathbb{Z}[x]$
 \end{itemize}
 \item 设$f(x)=a_nx^n+\cdots+a_0\in\mathbb{Z}[x]$为本原多项式.设$p$为素数,若$p\nmid a_0,p|a_1,\cdots,p| a_{n-1},p||a_n$,证明$f$在$\mathbb{Z}[x]$中不可约.
 \item 证明:
 \begin{itemize}
 \item[1)]设$f\in\mathbb{Z}[x]$为本原多项式,$p$为素数,若$f$的首项系数不被$p$整除,且$f\ mod\ p$在$\mathbb{F}_p[x]$中不可约,则$f$在$\mathbb{Z}[x]$中不可约.
 \item[2)]$x^4+x+1$在$\mathbb{F}_2[x]$中不可约.
 \item[3)]$x^4+3x+5$在$\mathbb{Z}[x]$中不可约. 
 \end{itemize}
 \item 设$n>1$是正整数,证明:如果$x^{n-1}+\cdots+x+1$在$\mathbb{Q}[x]$中不可约,则$n$是素数.
 \item 设$a_1,\cdots,a_n$是互不相同的整数,证明:$(x-a_1)\cdots(x-a_n)-1$在$\mathbb{Q}[x]$中不可约.\\({\color{blue}提示:若$(x-a_1)\cdots(x-a_n)-1=h(x)g(x)$,则$a_1,\cdots,a_n$为$h^2-1$和$g^2-1$的根})
 
 \item 对$f(x)\in \mathbb{Z}[x]$且$f(x)\neq 0$,用$c(f)$表示$f(x)$的容度.
 \begin{itemize}
 \item[(1)] 对任意$a\in \mathbb{Z},a\neq 0$,证明:$|c(af)|=|a\cdot c(f)|$;
 \item[(2)] 证明$|c(fg)|=|c(f)\cdot c(g)|$.
 \end{itemize}
 \item 设$f(x)$是本原多项式,$g(x)\in\mathbb{Q}[x]$,且$f(x)g(x)\in\mathbb{Z}[x]$,则$g(x)\in\mathbb{Z}[x]$.
 \item 设$p(x)\in \mathbb{Z}[x]$是本原的不可约多项式,证明:对$f
 (x),g(x)\in\mathbb{Z}[x]$,若$p(x)|f(x)g(x)$,则$p(x)|f(x)$或$p(x)|g(x)$.
 
 \end{enumerate}

{\color{red} 注意:12.17和12.19作业应于12.19-12.24期间提交}

\newpage
\head

\begin{center} % Everything within the center environment is centered.
	{\Large \bf 第二十七次作业} % <---- Don't forget to put in the right number
	\vspace{2mm}
	
        % YOUR NAMES GO HERE
	{\bf\quad   12/24 \quad  第十七周/星期二} % <---- Fill in your names here!
\end{center} 


\begin{enumerate}\setcounter{enumi}{184}
 \item 把置换$\sigma=(456)(567)(761)$写成不相交轮换的积
    \item 计算置换的乘积,并求乘积的阶:\begin{enumerate}
        \item[(1)] $[(135)(2467)]\cdot [(147)(2356)]$
        \item[(2)] $[(13)(57)(246)]\cdot [(135)(24)(67)]$
    \end{enumerate}
    \item 讨论置换$\sigma=\tbinom{1 \quad 2 \quad \cdots \quad n}{n \quad n-1 \quad \cdots \quad 1}$的奇偶性
    \item 证明$S_n(n\geq 3)$中的偶置换均为3轮换之积
    \item 证明$S_n$中奇置换的阶一定是偶数
    \item 证明$S_n$中型为$1^{\lambda_1}2^{\lambda_2}\cdots n^{\lambda_n}$的置换共有$\frac{n!}{\prod\limits_{i=1}^n \lambda_i !i^{\lambda_i} }$个.由此来证明:$$\sum\limits_{\lambda_i \geq 0,\lambda_1+2\lambda_2+\cdots+n\lambda_n=n} \frac{1}{\prod\limits_{i=1}^n \lambda_i !i^{\lambda_i}}=1$$.
    \item 当$n\geq 2$时,证明:$(12)$和$(123\cdots n)$是$S_n$的一组生成元.
\end{enumerate}

{\color{red} 注意:12.24和12.26作业应于12.26-12.31期间提交}


\newpage
\head

\begin{center} % Everything within the center environment is centered.
	{\Large \bf 第二十八次作业} % <---- Don't forget to put in the right number
	\vspace{2mm}
	
        % YOUR NAMES GO HERE
	{\bf\quad   12/26 \quad  第十七周/星期四} % <---- Fill in your names here!
\end{center} 


\begin{enumerate}\setcounter{enumi}{191}
    \item 设置换$\tbinom{1 \quad 2 \quad \cdots \quad n}{a_1 \quad a_2 \cdots \quad a_n}$的交错数为k,求置换$\tbinom{1 \quad 2 \quad \cdots \quad n}{a_n \quad a_{n-1} \cdots \quad a_1}$的交错数.
    \item 考虑$S_n$中置换$\sigma=\tbinom{1 \quad 2 \quad \cdots \quad n}{a_1 \quad a_2 \cdots \quad a_n}$,请问何时$\sigma$的交错数最大.
    \item 给定四元多项式$f$,令$G_f=\{\sigma \in S_4| (\sigma f)(x_1,x_2,x_3,x_4)=f(x_1,x_2,x_3,x_4)\}$.证明$G_f$是$S_4$的子群,并求下列给定$f$的$G_f$:\begin{enumerate}
        \item[(1)] $f=x_1x_2+x_3x_4$;
        \item[(2)] $f=x_1x_2x_3x_4$.
    \end{enumerate}
    \item 将下列对称多项式写成初等对称多项式的多项式:\begin{enumerate}
        \item[(1)] $x_1^2x_2+x_2^2x_1+x_1^2x_3+x_3^2x_1+x_2^2x_3+x_3^2x_2$;
        \item[(2)] $x_1(x_2^3+x_3^3)+x_2(x_1^3+x_3^3)+x_3(x_1^3+x_2^3)$.
    \end{enumerate}
    \item 试求$s_i(1,\zeta_n,\cdots,\zeta_n^{n-1})$,其中$s_i$为关于$x_1,\cdots,x_n$的i次初等对称多项式,$\zeta_n$为n次单位根.
    \item 取$\alpha,\beta \in S_n$,证明:\begin{enumerate}
        \item[(1)] $\alpha\beta\alpha^{-1}\beta^{-1} \in A_n$;
        \item[(2)] $\alpha\beta\alpha^{-1}\in A_n$当且仅当$\beta \in A_n$. 
    \end{enumerate}
    \item 对于正整数n,证明$x^n+x^{-n}$是关于$x+x^{-1}$的整系数多项式.
    \item 多项式$3x^3+2x^2-1$的根在$\mathbb{C}$上有三个不同的根,设为$r_1,r_2,r_3$.求多项式$f(x) \in \mathbb{Q}[x]$,使得它的根恰为$r_1^2,r_2^2,r_3^2$.
    \item({\color{red}选做})我们记$t_k=\sum\limits_{i=1}^{n}x_i^{k}(k\geq 1)$,特别地$t_0=n$,设$f(x)=\prod\limits_{i=1}^n(x-x_i)=x^n-s_1x^{n-1}+s_2x^{n-2}+\cdots+(-1)^ns_n$,证明下列等式:\begin{enumerate}
        \item 若$k\leq n-1$,则$t_k-t_{k-1}s_1+t_{k-2}s_2+\cdots+(-1)^{k-1}t_1s_{k-1}+(-1)^{k}ks_k=0$;
        \item 若$k\geq n$,则$t_k-t_{k-1}s_1+t_{k-2}s_2+\cdots+(-1)^nt_{k-n}s_n=0$
    \end{enumerate}
\end{enumerate}

{\color{red} 注意:12.24和12.26作业应于12.26-12.31期间提交}
 
\end{document}
